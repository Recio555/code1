\documentclass{article}
\usepackage{amsmath}

\begin{document}

\title{Uso de las Reglas de los Signos en la Suma y la Multiplicación}
\author{}
\date{}
\maketitle

En matemáticas, las reglas de los signos son fundamentales para operar con números enteros, ya que permiten determinar el signo de los resultados en las operaciones de suma y multiplicación. Estas reglas son esenciales para realizar cálculos correctos y para comprender cómo interactúan los números positivos y negativos en diversos contextos. A continuación, exploraremos cómo se aplican estas reglas en la suma y la multiplicación.

\section*{Reglas de los Signos en la Suma}

La suma de números enteros involucra tanto a números positivos como negativos. La clave para realizar una suma correctamente radica en entender cómo se combinan estos signos. A continuación, se presentan las reglas principales:

\begin{enumerate}
    \item \textbf{Suma de dos números positivos:} Si ambos números son positivos, el resultado siempre será positivo. El valor absoluto del resultado es la suma de los valores absolutos de los dos números.\\
    Ejemplo:  
    \[
    5 + 3 = 8
    \]

    \item \textbf{Suma de dos números negativos:} Si ambos números son negativos, el resultado será negativo. El valor absoluto del resultado es la suma de los valores absolutos de los dos números.\\
    Ejemplo:  
    \[
    -5 + (-3) = -8
    \]

    \item \textbf{Suma de un número positivo y uno negativo:} Cuando se suman un número positivo y un número negativo, se restan los valores absolutos de los números y el signo del resultado será el del número con mayor valor absoluto.\\
    Ejemplo 1:  
    \[
    7 + (-4) = 3
    \] (como \(7 > 4\), el resultado es positivo)\\
    Ejemplo 2:  
    \[
    -7 + 4 = -3
    \] (como \(7 > 4\), el resultado es negativo)
\end{enumerate}

\section*{Reglas de los Signos en la Multiplicación}

La multiplicación de números enteros sigue un conjunto específico de reglas para determinar el signo del resultado. En este caso, es importante considerar los signos de los números que estamos multiplicando.

\begin{enumerate}
    \item \textbf{Multiplicación de dos números positivos:} El resultado será siempre positivo. El producto de dos números positivos es positivo.\\
    Ejemplo:  
    \[
    4 \times 3 = 12
    \]

    \item \textbf{Multiplicación de dos números negativos:} El resultado será positivo. El producto de dos números negativos es positivo.\\
    Ejemplo:  
    \[
    (-4) \times (-3) = 12
    \]

    \item \textbf{Multiplicación de un número positivo y uno negativo:} El resultado será negativo. El producto de un número positivo y uno negativo es negativo.\\
    Ejemplo:  
    \[
    4 \times (-3) = -12
    \]

    \item \textbf{Multiplicación de un número negativo y uno positivo:} Similar al caso anterior, el resultado será negativo.\\
    Ejemplo:  
    \[
    (-4) \times 3 = -12
    \]
\end{enumerate}

\section*{Resumen}

Las reglas de los signos para la suma y la multiplicación son esenciales para resolver problemas matemáticos con números enteros. En resumen:

\begin{itemize}
    \item \textbf{Suma:}
    \begin{itemize}
        \item \( (+) + (+) = (+) \)
        \item \( (-) + (-) = (-) \)
        \item \( (+) + (-) \) o \( (-) + (+) \): Se restan los valores absolutos y el signo depende del número con mayor valor absoluto.
    \end{itemize}

    \item \textbf{Multiplicación:}
    \begin{itemize}
        \item \( (+) \times (+) = (+) \)
        \item \( (-) \times (-) = (+) \)
        \item \( (+) \times (-) = (-) \) o \( (-) \times (+) = (-) \)
    \end{itemize}
\end{itemize}

Comprender y aplicar correctamente estas reglas es crucial para resolver con éxito una amplia gama de problemas matemáticos.

\end{document}
