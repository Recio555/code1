\documentclass{article}
\usepackage{amsmath}
\usepackage{amsfonts}
\usepackage{geometry}
\geometry{a4paper}

\title{MHF4U Unidad 1 Rates of Change in Rational Functions Assessment Answers}
\author{}
\date{}

\begin{document}

\maketitle

\section*{Part 1: Multiple Choice Questions (Conceptual Understanding)}

\textbf{1. Which of the following is true about the average rate of change of a function between two points?}

\textbf{Answer}: A) It is the same as the slope of a secant line. \\
\textit{Explanation}: The **average rate of change** of a function between two points is equivalent to the **slope** of the secant line connecting these points.

\bigskip

\textbf{2. If a graph has a horizontal tangent line, what does this indicate about the instantaneous rate of change at that point?}

\textbf{Answer}: C) The instantaneous rate of change is zero. \\
\textit{Explanation}: A **horizontal tangent line** implies that the **slope** at that point is zero, which means the instantaneous rate of change is zero.

\bigskip

\textbf{3. In which of the following cases would the average rate of change be zero?}

\textbf{Answer}: D) A function that decreases and then increases. \\
\textit{Explanation}: The **average rate of change** between two points is zero when the function starts and ends at the same value, even if it increases and decreases in between.

\bigskip

\textbf{4. What is the average rate of change for the function \( f(x) = x^2 \) between \( x = 1 \) and \( x = 3 \)?}

\textbf{Answer}: B) 4 \\
\textit{Explanation}: The average rate of change is calculated using the formula:
\[
\frac{f(3) - f(1)}{3 - 1} = \frac{9 - 1}{2} = 4
\]

\newpage

\section*{Part 2: Short Answer (Conceptual and Problem Solving)}

\textbf{1. Define the instantaneous rate of change and explain how it differs from the average rate of change.}

\textbf{Answer}: 
\begin{itemize}
    \item \textbf{Instantaneous Rate of Change}: It refers to the rate of change at a specific point on the curve. It is the slope of the tangent line at that point.
    \item \textbf{Average Rate of Change}: It is the change in the value of the function over an interval divided by the length of that interval, representing the slope of the secant line.
\end{itemize}

\bigskip

\textbf{2. Given the following graph of a function \( f(x) \), estimate the instantaneous rate of change at \( x = 2 \) using secant lines.}

\textbf{Answer}: To estimate the instantaneous rate of change, draw secant lines near \( x = 2 \) (for example, from \( x = 1.9 \) to \( x = 2.1 \)), calculate the slope of these secant lines, and use this slope to approximate the instantaneous rate of change.

\bigskip

\textbf{3. For the function \( f(x) = x^2 \), calculate the average rate of change between \( x = -2 \) and \( x = 2 \). How does this differ from the instantaneous rate of change at \( x = 2 \)?}

\textbf{Answer}: 
\begin{itemize}
    \item \textbf{Average Rate of Change}: 
    \[
    \frac{f(2) - f(-2)}{2 - (-2)} = \frac{4 - 4}{4} = 0
    \]
    \item \textbf{Instantaneous Rate of Change at \( x = 2 \)}: The derivative of \( f(x) = x^2 \) is \( f'(x) = 2x \). At \( x = 2 \):
    \[
    f'(2) = 2(2) = 4
    \]
\end{itemize}

\bigskip

\textbf{4. You are driving a car and the speedometer shows a changing speed. Describe how you would interpret the instantaneous rate of change in this context.}

\textbf{Answer}: The **instantaneous rate of change** in this context represents the **car's speed** at a specific moment in time. It shows how fast the car is traveling at that exact point, as opposed to the **average speed** over a longer period.

\newpage

\section*{Part 3: Graphing and Operations with Functions}

\textbf{1. Sketch the graph of a function that increases at a constant rate, becomes constant, and then decreases at a constant rate. Indicate the rate of change at different intervals.}

\textbf{Answer}: The graph is a piecewise linear function with three sections:
\begin{itemize}
    \item **Increasing section**: A straight line with a positive slope.
    \item **Constant section**: A horizontal line (rate of change = 0).
    \item **Decreasing section**: A straight line with a negative slope.
\end{itemize}

\bigskip

\textbf{2. Given \( f(x) = x^3 \), graph the function and its rate of change (first derivative).}

\textbf{Answer}: 
\begin{itemize}
    \item The graph of \( f(x) = x^3 \) is a cubic function with a turning point at \( (0, 0) \).
    \item The rate of change (first derivative) is:
    \[
    f'(x) = 3x^2
    \]
    \item The graph of \( f'(x) = 3x^2 \) is a parabola that opens upwards, with a vertex at the origin.
\end{itemize}

\bigskip

\textbf{3. Estimate the instantaneous rate of change at \( x = 1 \) for the function \( f(x) = 2x^2 - 3x + 4 \) using a secant line.}

\textbf{Answer}: 
To estimate the instantaneous rate of change, calculate the slope of secant lines near \( x = 1 \). For example, calculate the slope of the secant line from \( x = 0.9 \) to \( x = 1.1 \), then find the average of these slopes.

\bigskip

\textbf{4. Graph the function \( y = 3x + 2 \) and the secant line between the points \( (1, f(1)) \) and \( (3, f(3)) \). Calculate the average rate of change.}

\textbf{Answer}: 
\[
\text{Average rate of change} = \frac{f(3) - f(1)}{3 - 1}
\]
For the function \( y = 3x + 2 \):
\[
f(3) = 3(3) + 2 = 11 \quad \text{and} \quad f(1) = 3(1) + 2 = 5
\]
\[
\frac{11 - 5}{3 - 1} = \frac{6}{2} = 3
\]
So, the average rate of change is **3**.

\newpage

\section*{Part 4: Word Problems (Real-World Applications)}

\textbf{1. A car accelerates from 0 to 60 km/h over a period of 5 seconds. What is the average rate of change of speed? What is the instantaneous rate of change at \( t = 5 \) seconds?}

\textbf{Answer}: 
\begin{itemize}
    \item Average rate of change:
    \[
    \frac{60 - 0}{5 - 0} = 12 \, \text{km/h per second}
    \]
    \item Instantaneous rate of change at \( t = 5 \) seconds depends on the function describing the car's acceleration. If the acceleration is constant, the instantaneous rate of change is the same as the average rate of change.
\end{itemize}

\bigskip

\textbf{2. A tank is being filled with water. The rate of change of water volume is modeled by \( V(t) = 4t^2 \), where \( t \) is time in hours. Find the instantaneous rate of change at \( t = 3 \) hours.}

\textbf{Answer}: 
\[
V'(t) = 8t
\]
At \( t = 3 \):
\[
V'(3) = 8(3) = 24 \, \text{cubic units per hour}
\]

\bigskip

\textbf{3. A company’s profit \( P(x) \) is modeled by the function \( P(x) = 100x - 2x^2 \), where \( x \) is the number of units sold. Calculate the instantaneous rate of change of profit when 10 units are sold.}

\textbf{Answer}: 
\[
P'(x) = 100 - 4x
\]
At \( x = 10 \):
\[
P'(10) = 100 - 4(10) = 60
\]
The instantaneous rate of change of profit when 10 units are sold is **60 units of profit per unit sold**.

\end{document}
