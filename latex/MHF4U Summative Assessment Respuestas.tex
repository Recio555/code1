\documentclass{article}
\usepackage{amsmath}
\usepackage{amsfonts}
\usepackage{geometry}
\geometry{a4paper}

\title{MHF4U - Rates of Change in Rational Functions Assessment Answers}
\author{}
\date{}

\begin{document}

\maketitle

\section*{Part 1: Multiple Choice Questions (Conceptual Understanding)}

\textbf{1. Which of the following is a valid operation on two functions, \( f(x) \) and \( g(x) \)?}

\textbf{Answer}: D) All of the above \\
\textit{Explanation}: All options represent valid operations on functions. You can add, multiply, and divide functions, and \( f(x) \cdot g(x) \) represents the product of the two functions.

\bigskip

\textbf{2. Which of the following is the correct domain for the function \( f(x) = \frac{1}{x - 2} \)?}

\textbf{Answer}: A) \( x \in (-\infty, 2) \cup (2, \infty) \) \\
\textit{Explanation}: The domain of the function is all real numbers except \( x = 2 \), since division by zero is undefined.

\bigskip

\textbf{3. Which transformation occurs when the absolute value function \( f(x) = |x| \) is modified to \( f(x) = |x - 3| \)?}

\textbf{Answer}: D) Horizontal shift right by 3 units \\
\textit{Explanation}: The modification \( |x - 3| \) shifts the function 3 units to the right along the x-axis.

\bigskip

\textbf{4. What is the range of the function \( f(x) = |x - 2| \)?}

\textbf{Answer}: D) \( [0, \infty) \) \\
\textit{Explanation}: The range of an absolute value function is always non-negative. Since the minimum value of \( |x - 2| \) is 0, the range is \( [0, \infty) \).

\newpage

\section*{Part 2: Short Answer (Conceptual and Problem Solving)}

\textbf{1. Define inverse functions. How can you verify if two functions are inverses of each other?}

\textbf{Answer}: 
\begin{itemize}
    \item \textbf{Inverse Functions}: Two functions \( f(x) \) and \( g(x) \) are said to be inverses of each other if \( f(g(x)) = x \) and \( g(f(x)) = x \).
    \item \textbf{Verification}: To verify if two functions are inverses, we compose them. If both compositions result in the identity function \( x \), then they are inverses.
\end{itemize}

\bigskip

\textbf{2. Given the piecewise function:}
\[
f(x) = 
\begin{cases} 
2x + 3 & \text{if } x < 0 \\
x^2 - 1 & \text{if } x \geq 0
\end{cases}
\]

\textbf{Domain}: All real numbers (\( \mathbb{R} \)), because the function is defined for all \( x \).

\textbf{Range}: 
\begin{itemize}
    \item For \( x < 0 \), the output of \( f(x) = 2x + 3 \) is all real numbers.
    \item For \( x \geq 0 \), the output of \( f(x) = x^2 - 1 \) is \( [-1, \infty) \).
\end{itemize}

Therefore, the range is \( (-\infty, \infty) \).

\bigskip

\textbf{3. Sketch the graph of this piecewise function.}

\textit{Explanation}: The graph consists of a line for \( x < 0 \) with slope 2 and y-intercept 3, and a parabola for \( x \geq 0 \) opening upwards starting from \( y = -1 \).

\bigskip

\textbf{4. Find the inverse of the function \( f(x) = 3x - 4 \).}

\textbf{Answer}: 
To find the inverse:
\[
y = 3x - 4
\]
Solve for \( x \):
\[
x = \frac{y + 4}{3}
\]
Thus, the inverse function is:
\[
f^{-1}(x) = \frac{x + 4}{3}
\]

\bigskip

\textbf{5. Solve the equation \( |2x - 3| = 7 \). Show all steps.}

\textbf{Answer}: 
\[
|2x - 3| = 7
\]
This gives two cases:
\begin{itemize}
    \item Case 1: \( 2x - 3 = 7 \) → \( 2x = 10 \) → \( x = 5 \)
    \item Case 2: \( 2x - 3 = -7 \) → \( 2x = -4 \) → \( x = -2 \)
\end{itemize}
Therefore, the solutions are \( x = 5 \) and \( x = -2 \).

\newpage

\section*{Part 3: Graphing and Operations with Functions}

\textbf{1. Sketch the graph of the following functions and indicate key features:}

\begin{itemize}
    \item \( f(x) = |x - 1| \): This is a V-shaped graph, with the vertex at \( (1, 0) \). The graph opens upwards.
    \item \( g(x) = 2x + 3 \): This is a straight line with slope 2 and y-intercept 3.
\end{itemize}

\bigskip

\textbf{2. Perform the following operations and graph the resulting functions:}

\[
h(x) = f(x) + g(x), \text{ where } f(x) = 2x - 1 \text{ and } g(x) = x^2 + 3x
\]

\textbf{Answer}: 
\[
h(x) = (2x - 1) + (x^2 + 3x) = x^2 + 5x - 1
\]
\textbf{Domain}: All real numbers (\( \mathbb{R} \)) \\
\textbf{Range}: \( (-\infty, \infty) \), since the function is a parabola opening upwards.

\bigskip

\textbf{3. Given \( f(x) = x^2 - 4 \) and \( g(x) = x + 1 \), find \( (f \circ g)(x) \).}

\textbf{Answer}: 
\[
(f \circ g)(x) = f(g(x)) = f(x + 1)
\]
Substituting \( x + 1 \) into \( f(x) \):
\[
f(x + 1) = (x + 1)^2 - 4 = x^2 + 2x + 1 - 4 = x^2 + 2x - 3
\]
Thus, \( (f \circ g)(x) = x^2 + 2x - 3 \).

\newpage

\section*{Part 4: Word Problems (Real-World Applications)}

\textbf{1. A population of rabbits grows exponentially according to the function \( P(t) = 50e^{0.05t} \), where \( t \) is time in years and \( P(t) \) is the population size.}

\textbf{How many rabbits are in the population after 10 years?}

\textbf{Answer}: 
\[
P(10) = 50e^{0.05 \times 10} = 50e^{0.5} \approx 50 \times 1.6487 \approx 82.44
\]
The population after 10 years is approximately 82 rabbits.

\bigskip

\textbf{What is the rate of change of the population at \( t = 10 \)?}

\textbf{Answer}: 
\[
P'(t) = 50e^{0.05t} \times 0.05
\]
At \( t = 10 \):
\[
P'(10) = 82.44 \times 0.05 \approx 4.12
\]
The rate of change at \( t = 10 \) is approximately 4.12 rabbits per year.

\bigskip

\textbf{2. A company’s profit is modeled by the function \( P(x) = 100x - 5x^2 \), where \( x \) is the number of units sold.}

\textbf{Find the number of units that maximizes profit.}

\textbf{Answer}: 
\[
P'(x) = 100 - 10x = 0 \quad \Rightarrow \quad x = 10
\]
The number of units that maximizes profit is 10 units.

\bigskip

\textbf{What is the maximum profit?}

\textbf{Answer}: 
\[
P(10) = 100(10) - 5(10)^2 = 1000 - 500 = 500
\]
The maximum profit is \$500.

\bigskip

\textbf{3. The cost to produce \( x \) units of a product is given by the function \( C(x) = 100 + 10x \), and the revenue function is \( R(x) = 50x \).}

\textbf{Determine the break-even point by solving \( C(x) = R(x) \).}

\textbf{Answer}: 
\[
100 + 10x = 50x \quad \Rightarrow \quad 100 = 40x \quad \Rightarrow \quad x = 2.5
\]
The break-even point is at 2.5 units.
\end{document}